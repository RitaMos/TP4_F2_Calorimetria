% ******************************
%* FILE CONFIGURATION
% ******************************
\documentclass[12pt, a4paper]{article}

\usepackage[spanish, es-tabla]{babel} % Enables Spanish language support and table naming
\usepackage{hyperref}                 % Enables creation of hyperlinks in the document
\usepackage{apacite}                  % Enables bibliography management
\usepackage{graphicx}                 % Enables inclusion of images and figures
\usepackage{multirow}                 % Provides multirow functionality for tables
\usepackage{float}                    % Provides better control for floating elements like tables and figures
\usepackage{fancyhdr}                 % Allows customization of headers and footers

% Sets page margins
\usepackage[left=2.5cm, right=2.5cm, top=2cm, bottom=2cm]{geometry}

\pagestyle{fancy}
\fancyhf{}                            % Clears all header and footer fields
\fancyfoot[C]{\thepage}               % Centers the page number at the footer
\renewcommand{\headrulewidth}{0pt}    % Removes the header line
\renewcommand{\footrulewidth}{0pt}    % Removes the footer line

\fancypagestyle{plain}                % Forces the same style on first page
{
  \fancyhf{}
  \fancyfoot[C]{\thepage}
  \renewcommand{\headrulewidth}{0pt}
  \renewcommand{\footrulewidth}{0pt}
}

\setlength{\arrayrulewidth}{0.5mm}    % Adjusts table bstep width
\setlength{\tabcolsep}{5pt}           % Adjusts spacing between table columns

\def\tablename{Tabla}                 % Changes the name of the table caption

% Creates a custom numbered step command (with bold numbers)
\newcounter{step}
\newcommand{\step}[1]
{
  \par\vspace{2ex}
  \stepcounter{step}
  \noindent\textbf{\arabic{step}.} #1\par\vspace{1ex}
}

% Creates a custon numbered step command (without bold numbers)
\newcounter{normalstep}
\newcommand{\normalstep}[1]
{
  \par\vspace{1ex}
  \stepcounter{normalstep}
  \noindent{\arabic{normalstep}.} #1\par\vspace{1ex}
}


% ******************************
%* PSEUDO-CARATULA
% ******************************

\title{TP 4: Calorimetría}
\author
{
  Caorsi Juan Ignacio, \href{jcaorsi@itba.edu.ar}{jcaorsi@itba.edu.ar} \\
  Dib Ian, \href{idib@itba.edu.ar}{idib@itba.edu.ar} \\
  Moschini Rita, \href{rmoschini@itba.edu.ar}{rmoschini@itba.edu.ar} \\
  Tamagnini Ana, \href{atamagnini@itba.edu.ar}{atamagnini@itba.edu.ar}
}

\date{Grupo 4 - 13/05/2025}

\begin{document}
\maketitle


% ******************************
%* SECCION 1
% ******************************
\step{Determinación de la capacidad calorífica del calorímetro}

Esta primera parte de la experiencia tuvo como objetivo la determinación de la capacidad calorífica del calorímetro. Para lograrlo, se buscó calcular su masa equivalente $\pi$, que representa la masa de agua que absorbería la misma cantidad de calor que absorbió el calorímetro.

El valor de $\pi$ se obtuvo mezclando dos masas de agua dentro del calorímetro: una masa $m_1$ a una temperatura $T_1$ alrededor de 10°C menor que la temperatura ambiente $T_0$, y otra masa $m_2$ alrededor de 20°C más caliente que la temperatura ambiente. Luego, se midió la temperatura de equilibrio $T_{eq}$ alcanzada dentro del calorímetro tras agitar. 

Los valores registrados se pueden observar en la siguiente tabla:

\begin{table}[h!]
\centering
    \begin{tabular}{|c|c|c|c|c|}
        \hline
            \textit{$m_1$ (g)} & \textit{$T_1$ (°C)} & \textit{$m_2$ (g)} & \textit{$T_2$ (°C)} & \textit{$T_{eq}$ (°C)} \\
        \hline
            159 & 13,3 & 153 & 36,6 & 24,6 \\
        \hline
    \end{tabular}
\caption{Mediciones registradas para la determinación de la capacidad calorífica del calorímetro}
\end{table}

Realizando el siguiente despeje se obtiene la masa equivalente:
\begin{equation}
    \Delta Q_1 + \Delta Q_{calorimetro} + \Delta Q_2 = 0
\end{equation}

\begin{equation}
    (m_1 + \pi) c_{agua} (T_{eq} - T_1) + m_2 c_{agua} (T_{eq} - T_2) = 0
\end{equation}

\begin{equation}
    \pi = \frac{m_2 (T_2 - T_{eq})}{T_{eq} - T_1} - m_1.
\end{equation}

\begin{center}
    $ \pi = 153g \cdot \frac{36,6 ^\circ C - 24,6 ^\circ C }{24,6 ^\circ C - 13,3 ^\circ C } - 159g $
\end{center}

\begin{center}
    $ \pi = 3,48 g $
\end{center}


% ******************************
%* SECCION 2
% ******************************
\step{Determinación del calor específico de un sólido}

En esta segunda parte de la experiencia, con el objetivo de determinar el calor específico de un sólido, se colocó una masa de agua $m_A$ a una temperatura $T_A$ (aproximadamente igual a la temperatura ambiente) dentro del calorímetro junto con un cuerpo de masa $m_C$ con capacidad calorífica desconocida, calentado a una temperatura $T_C$. Una vez alcanzado el equilibrio térmico dentro del calorímetro, se tomó la temperatura $T_{eq}$. Este procedimiento fue repetido para dos cuerpos.

\begin{table}[h!]
\centering
    \begin{tabular}{|c|c|c|c|c|c|}
        \hline
            & \textit{$m_A$ (g)} & \textit{$T_A$ (°C)} & \textit{$m_C$ (g)} & \textit{$T_C$ (°C)} & \textit{$T_{eq}$ (°C)} \\
        \hline
            cuerpo 1 & 132 & 24,5 & 84 & 101 & 28,6 \\
        \hline
            cuerpo 2 & 124 & 21,1 & 27 & 101 & 24,7 \\
        \hline
    \end{tabular}
\caption{Mediciones registradas para la determinación del calor específico de dos sólidos desconocidos.}
\end{table}


Al llegar al equilibrio térmico, conociendo el valor de $\pi$ y tomando la capacidad calorífica del agua $c_a =  1 \frac{cal}{g \cdot ^\circ C}$ se puede despejar $c_C$ como

\begin{equation}
    (m_A + \pi) c_a (T_{eq} - T_A) + m_C c_C (T_{eq} - T_C) = 0 \Longleftrightarrow
\end{equation}

\begin{equation}
    c_C = \frac{ - (m_A + \pi) \cdot c_A \cdot (T_{eq} - T_A) }{ m_C \cdot (T_{eq} - T_C) }
\end{equation}

Para el cuerpo 1, se tiene:

\begin{center}
    $ c_C = \frac{ - (132g + 3,478g) \cdot 1 \frac{cal}{g \cdot ^\circ C} \cdot (28,6^\circ C - 24,5^\circ C) }{ 84g \cdot (28,6^\circ C - 101^\circ C) } $
\end{center}
\begin{center}
    $ c_C = 0,0913 \frac{cal}{g \cdot ^\circ C} $
\end{center}

Comparando este valor experimental con los valores teóricos tabulados en \cite{educamix_capacidad, fisicanet_calor}, se observa que el resultado concuerda con el del bronce, cuyo valor teórico es de $ 0,086 \frac{cal}{g \cdot ^\circ C} $. De esta manera, el error experimental resulta:

\begin{center}
    $ E =  \frac{|0,0913-0,0860|}{|0,0860|}\cdot 100 = 6,2 \% $
\end{center}

Para el cuerpo 2, en cambio, se tiene:

\begin{equation}
    c_C = \frac{ - (124g + 3,478g) \cdot 11 \frac{cal}{g \cdot ^\circ C} \cdot (24,7^\circ C - 21,1^\circ C) }{ 27g \cdot (24,7^\circ C - 101^\circ C) }
\end{equation}
\begin{center}
    $ c_C = 0,223 \frac{cal}{g \cdot ^\circ C} $
\end{center}

El valor teórico más cercano encontrado en las fuentes consultadas \cite{educamix_capacidad,fisicanet_calor} fue $ 0,216 \frac{cal}{g \cdot ^\circ C} $, lo que sugiere que el cuerpo podría estar compuesto de aluminio. El error experimental asociado en este caso es:
\begin{center}
    $ E =  \frac{|0,216-0,223|}{|0,216|}\cdot 100 = 3,2 \% $
\end{center}

Dado que en ambos casos el error relativo es inferior al 10\%, las mediciones presentan una buena concordancia con los valores teóricos, lo que permite inferir con razonable certeza el material de los cuerpos analizados.


\newpage
\bibliographystyle{apacite}           % Follows APA style
\bibliography{bibliography} 

\end{document}
