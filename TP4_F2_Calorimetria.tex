% ******************************
%* FILE CONFIGURATION
% ******************************
\documentclass[12pt, a4paper]{article}

\usepackage[spanish, es-tabla]{babel} % Enables Spanish language support and table naming
\usepackage{hyperref}                 % Enables creation of hyperlinks in the document
\usepackage{apacite}                  % Enables bibliography management
\usepackage{graphicx}                 % Enables inclusion of images and figures
\usepackage{multirow}                 % Provides multirow functionality for tables
\usepackage{float}                    % Provides better control for floating elements like tables and figures
\usepackage{fancyhdr}                 % Allows customization of headers and footers

% Sets page margins
\usepackage[left=2.5cm, right=2.5cm, top=2cm, bottom=2cm]{geometry}

\pagestyle{fancy}
\fancyhf{}                            % Clears all header and footer fields
\fancyfoot[C]{\thepage}               % Centers the page number at the footer
\renewcommand{\headrulewidth}{0pt}    % Removes the header line
\renewcommand{\footrulewidth}{0pt}    % Removes the footer line

\fancypagestyle{plain}                % Forces the same style on first page
{
  \fancyhf{}
  \fancyfoot[C]{\thepage}
  \renewcommand{\headrulewidth}{0pt}
  \renewcommand{\footrulewidth}{0pt}
}

\setlength{\arrayrulewidth}{0.5mm}    % Adjusts table bstep width
\setlength{\tabcolsep}{5pt}           % Adjusts spacing between table columns

\def\tablename{Tabla}                 % Changes the name of the table caption

% Creates a custom numbered step command (with bold numbers)
\newcounter{step}
\newcommand{\step}[1]
{
  \par\vspace{2ex}
  \stepcounter{step}
  \noindent\textbf{\arabic{step}.} #1\par\vspace{1ex}
}

% Creates a custon numbered step command (without bold numbers)
\newcounter{normalstep}
\newcommand{\normalstep}[1]
{
  \par\vspace{1ex}
  \stepcounter{normalstep}
  \noindent{\arabic{normalstep}.} #1\par\vspace{1ex}
}


% ******************************
%* PSEUDO-CARATULA
% ******************************

\title{TP 4: Calorimetría}
\author
{
  Caorsi Juan Ignacio, \href{jcaorsi@itba.edu.ar}{jcaorsi@itba.edu.ar} \\
  Dib Ian, \href{idib@itba.edu.ar}{idib@itba.edu.ar} \\
  Moschini Rita, \href{rmoschini@itba.edu.ar}{rmoschini@itba.edu.ar} \\
  Tamagnini Ana, \href{atamagnini@itba.edu.ar}{atamagnini@itba.edu.ar}
}

\date{Grupo 4 - 13/05/2025}

\begin{document}
\maketitle

\begin{equation}
    T_{eq} = \frac{C_A T_A + C_B T_B}{C_A + C_B}
\end{equation}
C = mc
\begin{equation}
    \Delta Q = mc (T_f - T_i),
\end{equation}



% ******************************
%* SECCION 1
% ******************************
\section{Determinación de la Capacidad Calorífica del Calorímetro}
En la siguiente tabla,
\begin{itemize}
    \item $T_0$: temperatura ambiental
    \item $m_1$: masa de agua cuya temperatura es 10°C menor que $T_0$ previamente a colocarla en el calorímetro.
    \item $T_1$: temperatura de la masa $m_1$ medida dentro del calorímetro una vez alcanzado el equilibrio térmico.
    \item $m_2$: masa de agua cuya temperatura es 20°C mayor que $T_0$ previamente a colocarla en el calorímetro.
    \item $T_2$: temperatura de la masa $m_2$ de agua antes de colocarla en el calorímetro.
    \item $T_{eq}$: temperatura de equilibrio alcanzada dentro del calorímetro una vez colocadas y agitadas las masas $m_1$ y $m_2$.
\end{itemize}
\begin{table}[h!]
\centering
\begin{tabular}{|c|c|c|c|c|}
\hline
\textit{$m_1$ (g)} & \textit{$T_1$ (°C)} & \textit{$m_2$ (g)} & \textit{$T_2$ (°C)} & \textit{$T_{eq}$ (°C)} \\
\hline
& & & & \\
\hline
\end{tabular}
\caption{Tabla 1: Mediciones registradas para la determinación de la capacidad calorífica del calorímetro}
\end{table}

\begin{equation}
    \Delta Q = C \Delta T
\end{equation}

\begin{equation}
    (m_1 + \pi) c_{\text{agua}} (T_{eq} - T_1) + m_2 c_{\text{agua}} (T_{eq} - T_2) = 0,
\end{equation}

\noindent donde $\pi$ es el equivalente en agua del calorímetro, por lo que despejando tenemos que:

\begin{equation}
    \pi = \frac{m_2 (T_2 - T_{eq})}{T_{eq} - T_1} - m_1.
\end{equation}



% ******************************
%* SECCION 2
% ******************************
\section{Determinación del Calor Específico de un Sólido}

\begin{itemize}
    \item $m_A$: masa de agua.
    \item $T_A$: temperatura de la masa $m_A$, aproximadamente igual a la temperatura ambiente.
    \item $m_C$: masa del cuerpo que se sumerge en el agua $m_A$.
    \item $T_C$: temperatura a la que se calienta la masa $m_C$
    \item $T_{eq}$: COMPLETAR
\end{itemize}


\begin{table}[h!]
\centering
\begin{tabular}{|c|c|c|c|c|c|}
\hline
 & \textit{$m_A$ (g)} & \textit{$T_A$ (°C)} & \textit{$m_C$ (g)} & \textit{$T_C$ (°C)} & \textit{$T_{eq}$ (°C)} \\
\hline
cuerpo 1 & & & & & \\
\hline
cuerpo 2 & & & & & \\
\hline
\end{tabular}
\caption{Mediciones registradas para la determinación del calor específico de dos sólidos desconocidos.}
\end{table}

Al llegar al equilibrio térmico se tiene que:
\begin{equation}
    (m_A + \pi) c_a (T_{eq} - T_A) + m_C c_C (T_{eq} - T_C) = 0,
\end{equation}

\end{document}
