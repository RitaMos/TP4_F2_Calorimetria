% ******************************
%* FILE CONFIGURATION
% ******************************
\documentclass[12pt, a4paper]{article}

\usepackage[spanish, es-tabla]{babel} % Enables Spanish language support and table naming
\usepackage{hyperref}                 % Enables creation of hyperlinks in the document
\usepackage{apacite}                  % Enables bibliography management
\usepackage{graphicx}                 % Enables inclusion of images and figures
\usepackage{multirow}                 % Provides multirow functionality for tables
\usepackage{float}                    % Provides better control for floating elements like tables and figures
\usepackage{fancyhdr}                 % Allows customization of headers and footers

% Sets page margins
\usepackage[left=2.5cm, right=2.5cm, top=2cm, bottom=2cm]{geometry}

\pagestyle{fancy}
\fancyhf{}                            % Clears all header and footer fields
\fancyfoot[C]{\thepage}               % Centers the page number at the footer
\renewcommand{\headrulewidth}{0pt}    % Removes the header line
\renewcommand{\footrulewidth}{0pt}    % Removes the footer line

\fancypagestyle{plain}                % Forces the same style on first page
{
  \fancyhf{}
  \fancyfoot[C]{\thepage}
  \renewcommand{\headrulewidth}{0pt}
  \renewcommand{\footrulewidth}{0pt}
}

\setlength{\arrayrulewidth}{0.5mm}    % Adjusts table bstep width
\setlength{\tabcolsep}{5pt}           % Adjusts spacing between table columns

\def\tablename{Tabla}                 % Changes the name of the table caption

% Creates a custom numbered step command (with bold numbers)
\newcounter{step}
\newcommand{\step}[1]
{
  \par\vspace{2ex}
  \stepcounter{step}
  \noindent\textbf{\arabic{step}.} #1\par\vspace{1ex}
}

% Creates a custon numbered step command (without bold numbers)
\newcounter{normalstep}
\newcommand{\normalstep}[1]
{
  \par\vspace{1ex}
  \stepcounter{normalstep}
  \noindent{\arabic{normalstep}.} #1\par\vspace{1ex}
}


% ******************************
%* PSEUDO-CARATULA
% ******************************

\title{TP 4: Calorimetría}
\author
{
  Caorsi Juan Ignacio, \href{jcaorsi@itba.edu.ar}{jcaorsi@itba.edu.ar} \\
  Dib Ian, \href{idib@itba.edu.ar}{idib@itba.edu.ar} \\
  Moschini Rita, \href{rmoschini@itba.edu.ar}{rmoschini@itba.edu.ar} \\
  Tamagnini Ana, \href{atamagnini@itba.edu.ar}{atamagnini@itba.edu.ar}
}

\date{Grupo 4 - 13/05/2025}

\begin{document}
\maketitle


% ******************************
%* SECCION 1
% ******************************
\section{Determinación de la Capacidad Calorífica del Calorímetro}
Esta primera parte de la experiencia tiene como objetivo la determinación de la capacidad calorífica del calorímetro. Para lograrlo, se busca calcular su masa equivalente $\pi$, la cual representa cuál tendría que ser el valor de la masa del calorímetro si el mismo estuviera hecho de agua para que absorba la cantidad de calor que absorbió, es decir cuánta masa de agua absorbe la cantidad de calor que absorbió el calorímetro.
\par Se obtuvo este valor mezclando dos masas de agua dentro del calorímetro (una 10°C más fría que el ambiente y otra 20°C más caliente). Luego, se midió la temperatura de equilibrio. Los valores registrados se pueden observar en la tabla 1, donde
\begin{itemize}
    \item $T_0$: temperatura ambiental
    \item $m_1$: masa de agua cuya temperatura es 10°C menor que $T_0$ previamente a colocarla en el calorímetro.
    \item $T_1$: temperatura de la masa $m_1$ medida dentro del calorímetro una vez alcanzado el equilibrio térmico.
    \item $m_2$: masa de agua cuya temperatura es 20°C mayor que $T_0$ previamente a colocarla en el calorímetro.
    \item $T_2$: temperatura de la masa $m_2$ de agua antes de colocarla en el calorímetro.
    \item $T_{eq}$: temperatura de equilibrio alcanzada dentro del calorímetro una vez colocadas y agitadas las masas $m_1$ y $m_2$.
\end{itemize}

\begin{table}[h!]
\centering
\begin{tabular}{|c|c|c|c|c|}
\hline
\textit{$m_1$ (g)} & \textit{$T_1$ (°C)} & \textit{$m_2$ (g)} & \textit{$T_2$ (°C)} & \textit{$T_{eq}$ (°C)} \\
\hline
159 & 13,3 & 158 & 36,6 & 24,6 \\
\hline
\end{tabular}
\caption{Mediciones registradas para la determinación de la capacidad calorífica del calorímetro}
\end{table}

Realizando el siguiente despeje se obtiene la masa equivalente:
\begin{equation}
    \Delta Q_1 + \Delta Q_{calorimetro} + \Delta Q_2 = 0
\end{equation}

\begin{equation}
    (m_1 + \pi) c_{agua} (T_{eq} - T_1) + m_2 c_{agua} (T_{eq} - T_2) = 0
\end{equation}

\begin{equation}
    \pi = \frac{m_2 (T_2 - T_{eq})}{T_{eq} - T_1} - m_1.
\end{equation}

\begin{center}
    $ \pi = 153g \cdot \frac{36,6 ^\circ C - 24,6 ^\circ C }{24,6 ^\circ C - 13,3 ^\circ C } - 159g $
\end{center}

\begin{center}
    $ \pi = 3,478 g $
\end{center}



% ******************************
%* SECCION 2
% ******************************
\section{Determinación del Calor Específico de un Sólido}
Con el fin de determinar el calor específico de un sólido, se coloca una masa $m_A$ de agua dentro del calorímetro junto con el cuerpo de masa $m_C$ cuya capacidad calorífica se desea determinar; a este último se lo colocó completamente sumergido. Este procedimiento fue repetido para dos cuerpos.

\begin{itemize}
    \item $m_A$: masa de agua.
    \item $T_A$: temperatura de la masa $m_A$, aproximadamente igual a la temperatura ambiente.
    \item $m_C$: masa del cuerpo que se sumerge en el agua $m_A$.
    \item $T_C$: temperatura a la que se calienta la masa $m_C$
    \item $T_{eq}$: temperatura de equilibrio alcanzada dentro del calorímetro una vez colocadas y agitadas las masas $m_A$ y $m_C$.
\end{itemize}

\begin{table}[h!]
\centering
\begin{tabular}{|c|c|c|c|c|c|}
\hline
 & \textit{$m_A$ (g)} & \textit{$T_A$ (°C)} & \textit{$m_C$ (g)} & \textit{$T_C$ (°C)} & \textit{$T_{eq}$ (°C)} \\
\hline
cuerpo 1 & 132 & 24,5 & 84 & 101 & 28,6 \\
\hline
cuerpo 2 & 124 & 21,1 & 27 & 101 & 24,7 \\
\hline
\end{tabular}
\caption{Mediciones registradas para la determinación del calor específico de dos sólidos desconocidos.}
\end{table}


Al llegar al equilibrio térmico se tiene que
\begin{equation}
    (m_A + \pi) c_a (T_{eq} - T_A) + m_C c_C (T_{eq} - T_C) = 0
\end{equation}
Luego, despejando $c_C$
\begin{equation}
    c_C = \frac{ - (m_A + \pi) \cdot c_A \cdot (T_{eq} - T_A) }{ m_C \cdot (T_{eq} - T_C) }
\end{equation}

Para el cuerpo 1 (hecho de bronce):
\begin{center}
    $ c_C = \frac{ - (132g + 3,478g) \cdot 1 \frac{cal}{g \cdot ^\circ C} \cdot (28,6^\circ C - 24,5^\circ C) }{ 84g \cdot (28,6^\circ C - 101^\circ C) } $
\end{center}
\begin{center}
    $ c_C = 0,091334 \frac{cal}{g \cdot ^\circ C} $
\end{center}

Siendo $ 0,086 \frac{cal}{g \cdot ^\circ C} $ el valor teórico, el error relativo resulta
\begin{center}
    $ E_r =  \frac{|0,086-0,091334|}{|0,091334|}\cdot 100 = 6,2 \% $
\end{center}

Para el cuerpo 2 (hecho de aluminio):
\begin{equation}
    c_C = \frac{ - (124g + 3,478g) \cdot 11 \frac{cal}{g \cdot ^\circ C} \cdot (24,7^\circ C - 21,1^\circ C) }{ 27g \cdot (24,7^\circ C - 101^\circ C) }
\end{equation}
\begin{center}
    $ c_C = 0,222766 \frac{cal}{g \cdot ^\circ C} $
\end{center}
Siendo $ 0,216 \frac{cal}{g \cdot ^\circ C} $ el valor teórico, el error relativo resulta
\begin{center}
    $ E_r =  \frac{|0,216-0,222766|}{|0,216|}\cdot 100 = 3,13 \% $
\end{center}

Dado que ambos errores son menores al 10\%, se considera que las mediciones son buenas y que podría deducirse el material de los cuerpos a partir de las mismas.

\bibliographystyle{apacite}           % Follows APA style
\bibliography{bibliography} 
Valores teóricos obtenidos de este \href{https://www.educamix.com/educacion/3_eso_materiales/prof/bloque_ii/tablas_d_te_tf_internet.pdf}{enlace} (hacer click en la palabra enlace).

\end{document}
